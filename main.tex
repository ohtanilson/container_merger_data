\documentclass[10pt]{article}
\usepackage[utf8]{inputenc}
\usepackage{amsmath,setspace,geometry}
\usepackage{amsfonts}
\usepackage[shortlabels]{enumitem}
%\usepackage[dvipdfmx]{hyperref,graphicx}
\usepackage{graphicx}
\usepackage{bbm}

\usepackage[colorlinks,citecolor=purple,urlcolor=blue,bookmarks=false,hypertexnames=true]{hyperref}
\usepackage[]{natbib} 
\bibpunct[:]{(}{)}{,}{a}{}{,}
\geometry{left = 1.0in,right = 1.0in,top = 1.0in,bottom = 1.0in}
%\onehalfspacing
% \usepackage{setspace}
\doublespacing
%\renewcommand{\baselinestretch}{0.3}
\usepackage[english]{babel}
\usepackage{float}
\usepackage{subfig}
\usepackage{booktabs}
\usepackage{pdfpages}
\usepackage{threeparttable}
\usepackage{lscape}
%\setstretch{1.2}
\newtheorem{assumption}{Assumption}
\newtheorem{definition}{Definition}
\newtheorem{example}{Example}
\newtheorem{lemma}{Lemma}

\title{Unified Merger List in the Container Shipping Industry from 1966: Structural Estimation of Transition of Importance of Firm's Age, Tonnage Capacity, and Geographical Proximity on Merger Decision}
\author{Suguru Otani\thanks{Department of Economics, Rice University. Email: so19@rice.edu}\quad  Takuma Matsuda\thanks{Faculty of Commerce, Takushoku University. Email: tmatsuda@ner.takushoku-u.ac.jp}}
\date{
First version: October 14, 2023\\
Current version: \today
}

\begin{document}

\maketitle

\begin{abstract}
We construct a novel unified list of mergers in the global container shipping industry between 1966 (the beginning of the industry) and 2022. Combining the list with proprietary data, we construct a structural matching model to describe the historical transition of the importance of firm's age, size, and geographical proximity on merger decisions. 
We find that, as a positive factor, firm's size is more important than firm's age by 9.974 times as a merger incentive between 1991 and 2005.
However, between 2006 and 2022, as a negative factor, firm's size is more important than firm's age by 0.026-0.630 times, that is, firm's size works as a disincentive.
We also find that the distance between buyer and seller firms works as a disincentive for the whole period, but the importance has dwindled to economic insignificance in recent years. 
In counterfactual simulations, we find that the prohibition of mergers between firms in the same country affects the merger configuration of not only firms involved in prohibited mergers, but also firms involved in permitted mergers.
\end{abstract} 

\vspace{0.1in}
\noindent\textbf{Keywords:} container shipping industry; competition policy; merger; matching 
\vspace{0in}


\section{Introduction}

Container shipping plays a pivotal role in global trade, revolutionizing the world. According to data from IHS Markit and Descartes Datamyne, it constituted 45.4\% of amount-based imports to the U.S., 21.3\% of amount-based exports from the U.S., and 10.1\% of quantity-based world trade in 2021.
Moreover, the container shipping industry presents an intriguing opportunity to investigate industry dynamics, including entry, exit, and investment \citep{otani2023industry}, as well as the history of mergers since its global shipping operations inception in 1966.
Despite its significance, there exists a notable absence of a consolidated dataset for container shipping mergers, particularly from 1966 to 1990, hindering quantitative research in this area. 
This study addresses this gap by providing a unified list of all realized mergers in the container shipping industry spanning from 1966 to 2022.

Using our new merger list, we initially depict the merger waves in the global container shipping industry from 1966 to 2022, comparing them to the price and quantity transitions constructed by \cite{matsuda2022unified}.
We observe that the first merger waves emerged following the enactment of the Shipping Act of 1984. 
Subsequently, significant merger waves occurred after 2005, aligning with the exponential growth in quantities under competitive prices. With targeted periods of the data sources, this observation leads us to categorize the industry's history into three distinct ``regimes" corresponding to the data: 1966-1990, 1991-2005, and 2006-2022.
Additionally, by merging our merger list with proprietary ship-level data, we delineate the merger patterns, revealing a tendency for mergers to involve relatively younger and smaller firms in more distant countries in recent years.
For example, Taiwan's Cheng Lie Navigation, acquired by France's CMA-CGM in 2006, was a relatively small container shipping company founded in 1971. 
Safmarine, acquired by Denmark's Maersk in 1999, was a South African shipping company. 
The data patterns are consistent with the view that, in addition to mergers and acquisitions, the market share of independent operators in the liner shipping industry has been declining in recent years \citep{Merk2022MEL}. 

The merger pattern provides crucial insights into the global container shipping industry's merger waves.
However, multiple factors can account for this pattern. 
For instance, recent acquisitions may prioritize firm's size to expand market tonnage shares, unlike the period from 1966 to 1990. 
Alternatively, companies may emphasize geographical proximity to establish dominance at the local level.
To untangle these explanations and gain more precise insights into various channels, we employ a structural model that quantifies the relative significance of firm's age, size, and geographical proximity using a novel approach — the matching maximum score estimator \cite{fox2018qe}.

Our estimation results indicate that assortativeness of both size and geographical proximity contributes to merger incentives or disincentives. 
First, the assortativeness of firm size shifts from negative (1991-2005) to positive (2006-2022). 
During 1991-2005, the importance of firm's size supersedes the importance of its age by a factor of 9.974, serving as a merger incentive.
Conversely, between 2006 and 2022, as a negative factor, firm's size is more important than firm's age by 0.026-0.630 times, that is, firm's size works as a disincentive.
Additionally, we observe that geographical distance acts as a merger disincentive throughout the entire period, albeit its relative importance compared to firm's age has dwindled to economic insignificance in recent years. 
This suggests diminished merger incentives for establishing dominance at the local country level.

Finally, we conduct a counterfactual simulation based on the estimated parameters to examine the consequences of prohibiting mergers between firms within the same country.
This type of merger restriction is highly contentious within the global market's competition policies, in particular, in global container shipping..
For instance, on June 21, 2017, South Africa's Competition Commission issued a statement forbidding the consolidation of container businesses by three shipping lines — Nippon Yusen Kaisha(NYK), Mitsui O.S.K. Lines (MOL), and Kawasaki Kisen Kaisha (KLINE). 
The commission expressed concerns about market consolidation by domestic companies and cartel issues related to these firms in the car carrier business.
Although the country's competition court eventually approved the integration on January 17, 2018, its potential impact on the planned launch of the integrated container company, Ocean Network Express, remains noteworthy.
In our counterfactual simulations, we discover that prohibiting mergers between firms in the same country affects not only the merger outcomes of the involved firms but also those engaged in permitted mergers. 
This foretells the ripple effect of local competition policies through equilibrium matchings, influencing both local markets and the global market configuration.

\subsection{Related literature}

This paper contributes to three strands of the literature, namely, empirical transferable utility (TU) matching, endogenous merger analysis, and recent industrial policy and antitrust studies in the shipping industry.

First, this paper contributes to the literature on empirical TU matching. 
See the recent methodological development in \cite{agarwal2021market}.
The most related econometric model is \cite{fox2010qe,fox2018qe}, whose model has been applied to other empirical topics such as banking merger \citep{akkus2015ms,chen2013ijio}, faculty room allocation \citep{baccara2012aer}, executive and firm matching \citep{pan2017determinants}, and buyer and seller relationships in the broadcast television industry \citep{stahl2016aer}. 
These papers have applied the matching maximum score estimator proposed by \cite{fox2010qe,fox2018qe} to two-sided many-to-many and one-to-one matching in a TU matching environment. 
We apply the approach to merger waves in the global container shipping industry from its inception by dividing the history into three regimes based on institutional knowledge and data period.
As far as we know, our paper is the first paper to illustrate historical transitions of assortativeness of observed variables using long panel data.

Second, this paper contributes to the literature on endogenous merger analysis. 
Endogenous merger analysis in the industrial organization literature is divided into dynamic and static matching models. 
In terms of dynamic matching models, they follow \cite{gowrisankaran1999dynamic}.\footnote{\cite{stahl2011dynamic} was the first to estimate a merger activity model using a dynamic, strategic framework. \cite{jeziorski2014effects} estimated the sequential merger process to analyze ownership consolidation in the United States radio industry after the enactment of the Telecommunications Act of 1996. \cite{igami2019mergers} applied a stochastic sequential bargaining model to the merger processes of the hard disk industry. As the most recent paper, \cite{hollenbeck2020horizontal} enriched the Gowrisankaran-type dynamic endogenous merger model. With different dynamic approaches, \cite{nishida2015better} compared post-merger and pre-merger beliefs and equilibrium behaviors in a Markov perfect equilibrium in the Japanese retail chain industry. \cite{perez2015building} incorporated mergers as bidding games by incumbents and investigated the effect of the Reagan-Bush administration's merger policy on the reallocation of assets in the United States cement industry.} Conversely, using a static matching model, \cite{uetake2019entry} developed an empirical two-sided non-transferred utility matching model with externalities using moment inequalities and investigated the effect of entry deregulation on the ``with whom"-decisions of bank mergers by the Riegle-Neal Act. 
\cite{akkus2015ms} tackled the same Act with a different approach. 
They added transfer data and constructed a one-to-one matching model with transfer utility and found that merger value increased from cost efficiencies in overlapping markets, relaxing regulations, and the network effects exhibited by acquirer-target matching. 
Our paper follows \cite{akkus2015ms} and focuses on endogenous mergers in a single, static, large matching market for each regime and quantifies the relative importance of tonnage capacity and geographical proximity, which are the main economic forces driving firms to pursue mergers to gain cost efficiency in the shipping industry \citep{notteboom2004container}. 
In addition, we compare historical transitions of the relative importance of the variables to derive potentially different merger incentives behind merger waves in the industry.

Third, our paper contributes to the literature on recent industrial policy and antitrust studies in the shipping industry. In the industrial organization literature, \cite{jeon2022learning} studies the relationship between learning and investment in the container shipping industry between 2006 and 2014 and simulates social welfare in counterfactual merger scenarios in which a merger occurred between top two firms that jointly account for over 35\% of total capacity in the industry.
In the maritime shipping literature, there are various researches using the Herfindahl-Hirschman Index (HHI) and its modification, although empirical studies using simple regressions of HHI are criticized \citep{bresnahan1989empirical}.\footnote{As a theoretical merger analysis, \cite{nocke2022concentration} demonstrate that in a general Cournot model, only the naively-computed change in the HHI due to a merger (twice the product of the per-merger market shares of the merging firms), but not the level of the HHI, is valuable in screening mergers for welfare assessment. Out of the merger analysis, \cite{spiegel2021herfindahl} shows that HHI reflects the ratio of producer surplus to consumer surplus in Cournot markets under some theoretical conditions. The empirical maritime literature follows the direction of \cite{spiegel2021herfindahl} focusing on the relationship between market concentration, prices, and consumer surplus without taking care of the endogeneity problem on prices, HHI, and so on, for example, \cite{Hirata2017}. This has been criticized in the industrial organization literature \citep{bresnahan1989empirical} since the 1980s. HHI is still sometimes used by policymakers and practitioners to measure of market concentration of container shipping market. For example, the Federal Maritime Commission (FMC), the administrative agency that oversees shipping in the United States, cited the competitive nature of the HHI as one of a reason why Transpacific route is extremely competitive \citep{FMC2022}} For example,
\cite{Sys2009TransPOL} collects firm-year-level vessel volume data for the period 1999-2009 and calculates the HHI and Gini coefficients to compare the degree of market concentration. 
The author shows that while the degree of concentration has increased in years when mergers and acquisitions have taken place, the industry is still fragmented and competitive due to small shares of firms.
\cite{Merk2022MEL} uses a modified Herfindahl–Hirschman Index (MHHI) to show that industry concentration is higher when consortia are taken into account.
Although these researches treat specific hypothetical mergers and consortia as exogenously determined scenarios and focus on the post-merger market outcomes such as some welfare and concentration measures based on non-cooperative game theoretical models, our paper endogenizes mergers based on cooperative game-theoretical models, i.e., matching models, and focus on merger incentives. 
Our approach enables us to simulate hypothetical endogenous mergers based on inferred merger incentives instead of not being able to assess welfare and concentration measures as in the above papers.

The remainder of this paper is organized as follows. 
Section \ref{sec:data_and_institutional_background} summarizes the data and institutional background of mergers in the container shipping industry.
Section \ref{sec:empirical_analysis} constructs a structural matching model to quantify the assortativeness of observed characteristics for each regime and compare the levels across regimes.
Section \ref{sec:results} shows estimation results.
Section \ref{sec:counterfactuals} shows counterfactual simulation results.
%\textcolor{blue}{Section \ref{sec:interviews} provides interviews with XXX.}
Section \ref{sec:practical_implications} summarizes practical implications, discussion, and future research directions.
Finally, Section \ref{sec:conclusion} presents our conclusions.


\section{Data and Industry Background}\label{sec:data_and_institutional_background}
We provide details of the data source in Section \ref{sec:data_source}. 
Next, we provide industry background in
Section \ref{sec:industry_background} and summary statistics for the variables in Section \ref{sec:descriptive_statistics}.




\subsection{Data source}\label{sec:data_source}
We compile data by merging three distinct sources.
The initial source is the Containerization International Yearbook (CIY), offering ship-level information from 1966 to 1990. 
The second source is IHS Markit data (IHS), which provides ship-level information spanning 1991 to 2005. 
The third source, the Handbook of Ocean Commerce (HB), furnishes ship-level data covering the years 2006 to 2022.
We classify these periods in the respective data sources as ``regimes," resulting in three regimes: 1966-1990, 1991-2005, and 2006-2022.
By consolidating ship-level data, we construct firm-year-level variables, including country names and tonnage capacity measured in Twenty-foot Equivalent Units (TEU).
Finally, we manually create a merger list containing buyer and seller names along with merger years. 
This list is subsequently integrated with the firm-year-level variables using institutional information. 
Note that MDS Transmodal data is a potential alternative for a fee, but it offers a maximum of two years of panel raw data, which includes ship-year level variables used in our analysis.
Thus, we believe that our data construction is conducted in the best and most feasible way.

There are some remarks because we find that there are some inconsistencies such as a one-year lag and missing ship-level variables between these data sources and institutional facts.
First, we fix the inconsistencies following the observations in the newer regime data. 
Second, we treat the firms not operating in the merged year as the firms that have a constant capacity level from the last active year in the merged year. % For example, Johnson Line was active in 1969-1972 but was merged in 1991.
Third, we treat mergers of container shipping seller firms by non-container-shipping firms outside the industry as exits from the container shipping sector, as these mergers lack information on buyer firms.
Fourth, we treat consolidation-type mergers as mergers in which buyer firms have the lower bound of age and size variables at the initial merger timing.
The final data regarding mergers is summarized in this section and used in empirical analysis in Section \ref{sec:empirical_analysis}. 



Figure \ref{fg:number_of_mergers} illustrates the number of mergers between 1966 and 2022 based on our merger list.
For comparison, Figure \ref{fg:container_freight_rate_and_shipping_quantity_each_route} illustrates the trends in route-year-level shipping prices and quantities between 1966 and 2009.
Comparing these figures provides graphical intuition. 
First, Merger waves emerged after the enactment of the Shipping Act of 1984, signaling a shift from collusive behaviors.
Second, subsequent merger waves align with the exponential growth in quantities under competitive prices post-2005.
As a result, we categorize the industry's history into three distinct ``regimes": 1966-1990, 1991-2005, and 2006-2022, in accordance with the corresponding data.


\begin{figure}[!ht]
\begin{center}
  \includegraphics[width = 0.7\textwidth]
  {figuretable/number_of_mergers.png}
  \caption{The number of mergers between 1966 and 2022}
  \label{fg:number_of_mergers}
  \end{center}
\footnotesize
   Note: Red lines divide the regimes based on CIY, IHS, and HB.
\end{figure}

\begin{figure}[!ht]
\begin{center}
  \subfloat[Price]{\includegraphics[width = 0.7\textwidth]
  {figuretable/container_freight_rate_each_route.png}}\\
  \subfloat[Quantity]{\includegraphics[width = 0.7\textwidth]
  {figuretable/container_shipping_quantity_each_route.png}}
  \caption{Trends in route-year-level shipping prices and quantities.}
  \label{fg:container_freight_rate_and_shipping_quantity_each_route}
  \end{center}
\footnotesize
  Note: Prices are adjusted to the CPI in the U.S. in 1995. See the detail in \cite{matsuda2022unified}.
\end{figure}

\subsection{Industry Background}\label{sec:industry_background}
We describe the industry background between 1966 and 2022 chronologically by focusing on firms' mergers. 
As mentioned earlier, we divide the entire period into three distinct regimes—1966-1990, 1991-2005, and 2006-2022—aligned with the institutional background and data sources.
In the subsequent merger lists, we deliberately retain the original firm names for each regime's data, despite potential inconsistencies across these datasets.

\begin{table}[!htbp]
  \begin{center}
      \caption{Merger list: CIY (1966-1990)}
      \label{tb:merger_list_CIY} 
      
\begin{tabular}[t]{lrrl}
\toprule
Operator & Start & End & Merging firm\\
\midrule
Moore-McCormack Lines Inc & 1966 & 1970 & United States Lines\\
OCL & 1972 & 1986 & P&O Containers\\
Y-S Line & 1974 & 1988 & NLS\\
Atlanttrafik/Barber Blue Sea & 1974 & 1990 & Wilhelmsen Lines A/S\\
Franco-Belgian Services & 1976 & 1986 & Maersk\\
KSC & 1976 & 1988 & Hanjin\\
Finland Steamship & 1977 & 1990 & Finnlines\\
Japan Line & 1981 & 1988 & NLS\\
\bottomrule
\end{tabular}

  \end{center}\footnotesize
  %Note: 
\end{table} 

\begin{table}[!htbp]
  \begin{center}
      \caption{Merger list: IHS (1991-2005)}
      \label{tb:merger_list_IHS} 
      
\begin{tabular}[t]{rllr}
\toprule
ID & Seller & Buyer & Year\\
\midrule
1 & IMC SHIPPING CO PTE LTD & IMC SHIPPING CO PTE LTD & 1993\\
2 & BUSAN SHIPPING CO LTD & EUROSEAS LTD & 1994\\
3 & CHINA MERCHANTS STEAM NAVIGATI & China Merchants Group & 1994\\
4 & SVITZER AS & A P MOLLER & 1996\\
5 & APL LTD & NEPTUNE ORIENT LINES LTD (NOL) & 1997\\
6 & PRIMA SHIPMANAGEMENT SDN BHD & HALIM MAZMIN GROUP & 1999\\
7 & FARRELL LINES INC & A P MOLLER & 2000\\
8 & OOST ATLANTIC LIJN BV & ATLANTIC HORIZON GROUP & 2001\\
9 & CYPRUS MARITIME CO LTD & CYPRUS SEA LINES SA & 2002\\
10 & MISC BERHAD & Malaysia Shipping Corp Sdn Bhd & 2003\\
11 & DANSK SUPERMARKED INVEST A/S & A P MOLLER & 2003\\
12 & THE PENINSULAR AND ORIENTAL ST & A P MOLLER & 2004\\
13 & EUROBULK LTD & EUROSEAS LTD & 2005\\
14 & BARCLAY SHIPPING LTD & BARCLAY SHIPPING LTD & 2005\\
15 & DELMAS & CMA CGM HOLDING & 2005\\
16 & ROYAL P\&O NEDLLOYD NV & A P MOLLER & 2005\\
17 & UNITED THAI SHIPPING CORP LTD & IMC SHIPPING CO PTE LTD & 2005\\
18 & HORIZON LINES INC & MATSON NAVIGATION CO INC & 2005\\
19 & EICKE SCHIFFAHRTS KG & EICKE SCHIFFAHRTS KG & 2005\\
20 & CP SHIPS LTD & HAPAG-LLOYD AG & 2005\\
\bottomrule
\end{tabular}

  \end{center}\footnotesize
  Note: We omit a merger of Royal Nedlloyd and the Peninsular and Oriental containers (P\&O Containers) in 1997, Maersk and Sea-Land in 1999, and CMA and CGM in 1999 because historical vessel data of merged firm are missing and we could not identify these mergers from our data.
\end{table} 

\begin{table}[!htbp]
  \begin{center}
      \caption{Merger list: HB (2006-2022)}
      \label{tb:merger_list_HB} 
      \subfloat[HB (2006-2022)]{
\begin{tabular}[t]{rllrl}
\toprule
ID & Seller & Buyer & Year & Type\\
\midrule
1 & Cheng Lie & CMA-CGM & 2006 & acquisition\\
2 & Lloyd Triestino & Evergreen & 2006 & merger\\
3 & Norasia & CSAV & 2006 & acquisition\\
4 & MacAndrews & CMA-CGM & 2007 & acquisition\\
5 & Lufeng & Sinotrans & 2008 & merger\\
6 & NEW ONTO SHIPPING & GOTO Shipping International Ltd & 2010 & merger\\
7 & TSK & NYK & 2010 & merger\\
8 & China Navigation & Swire & 2011 & acquisition\\
9 & CCNI & Maersk & 2015 & acquisition\\
10 & CSAV & Hapag-Lloyd & 2015 & acquisition\\
11 & China Shipping & COSCO & 2016 & merger\\
12 & Shanghai Puhai Shipping & COSCO & 2016 & merger\\
13 & UASC & Hapag-Lloyd & 2017 & acquisition\\
14 & KLINE & Ocean Network Express & 2018 & merger\\
15 & MOL & Ocean Network Express & 2018 & merger\\
16 & NYK & Ocean Network Express & 2018 & merger\\
17 & APL & CMA-CGM & 2017 & acquisition\\
18 & Hamburg Sud & Maersk & 2018 & acquisition\\
\bottomrule
\end{tabular}
}\\
      \subfloat[HB (2006-2022): Inconsistent merger cases]{
\begin{tabular}[t]{rllrl}
\toprule
ID & Seller & Buyer & Year & Note\\
\midrule
1 & Safmarine & Maersk & 2008 & Merger occurred in 1999\\
2 & Delmas & CMA-CGM & 2016 & Merger occurred in 2005\\
3 & ANL & CMA-CGM & 2022 & Merger occurred in 1998\\
\bottomrule
\end{tabular}
}
  \end{center}\footnotesize
  Note: Panel (b) provides merger cases that have inconsistencies with the institutional background because ships operated by merged firms can keep the past operator name after merger years. Also, we omit Cosco's merger of OOCL in 2018 because merged OOCL's vessels keep OOCL as the registered operator name in the HB data until 2022, so we could not identify the renewal timing of the registered operator names.
\end{table} 

\paragraph{1966-1990} 

Table \ref{tb:merger_list_CIY} summarizes all mergers based on CIY between 1966 and 1990.\footnote{In 1964, the Japanese ocean shipping industry experienced consolidation induced by the government, and 95 firms were merged into six large groups. \cite{otani2021estimating} investigates the event by a structural matching model.}
This period involves a collusive and competitive environment with shipping conferences that are explicit and cartels globally allowed. 
The period is studied in \cite{matsuda2022unified} and \cite{otani2023industry} in detail.
The period is divided into the collusive (1966-1983) and competitive (1984-1990) periods due to the U.S. Shipping Act of 1984.\footnote{The relevant studies are \cite{wilson1991some}, \cite{pirrong1992application}, and \cite{clyde1998market}. \cite{wilson1991some} provided evidence of regime change by the Shipping Act of 1984 using data on quarterly freight rates and shipping quantities of five selected commodities only on the Transpacific route. \cite{pirrong1992application} tested the model prediction of the core theory surveyed in \cite{sjostrom2013competition} using data from two specific trade routes between 1983 and 1985. \cite{clyde1998market} studied the relationship between market power and the market share of shipping conferences after the act. }
In the collusive period between 1966 and 1983, a single merger occurred in 1970 (Moore-McCormack Lines Inc merged by United States Lines). % and 1972 (Johnson Line merged by NA).
In the competitive period between 1984 and 1990, two mergers occurred in 1986, three mergers occurred in 1988, and two mergers occurred in 1990. 

Until the 1970s, containerization still were not much adopted and pioneer countries and ports had set containerized operations only in North America, Western Europe, and Japan \citep{GUERRERO2014151}, and had dominant market shares. Also, container freight rates were high until the 1970s due to the shipping conferences, as shown in Figure \ref{fg:container_freight_rate_and_shipping_quantity_each_route}.
Thus, except for the Moore-McCormack merger, there were not significant mergers in the industry in the 1970s because collusive firms enjoyed enough profits without mergers. Institutionally, Japanese shipping executives stated in newspaper interviews that liner shipping, including containers, was the most profitable division in their companies at that time and supported their operations \citep{sato2006}. 

The 1980s in the industry is known as the expansion of containerization, which took place in North America, Western Europe, and Japan and its trading partners such as the Caribbean, the Mediterranean, and Southeast Asian countries \citep{GUERRERO2014151}. During this time, these areas were integrated into global trade relations through the beginning of offshoring and the emergence of new transshipment hubs such as Singapore. However, there was a significant decline in container freight rates due to the Sea-Land's withdrawal from shipping conferences in 1980 and the enactment of the Shipping Act of 1984 in the United States \citep{matsuda2022unified}. The falling freight rates and the depreciation of the dollar as a result of the Plaza Accord of 1985 affected the profitability of Japanese and European shipping companies and prompted the mergers \citep{Duru2018}. In Japan, Yamashita Shin Nihon Kisen Kaisha (Y-S Line) and Japan Line merged to form Nippon Liner System (NLS) in the container shipping sector. In 1988, Hanjin Shipping merged with Korea Shipping Company (KSC), which was originally a state-owned company. %As for Japanese shipping companies, the appreciation of the yen that occurred after 1985 was considered to have been a significant factor in mergers. 
% Background
% What mergers?

\paragraph{1991-2008}

Table \ref{tb:merger_list_IHS} summarizes all mergers based on IHS between 1991 and 2005.
The period involves the Ocean Shipping Reform Act (OSRA) of 1998 and this enactment divides the period into the pre-OSRA and post-OSRA periods. 
The period is studied in \cite{fusillo2006some,fusillo2013stability} and \cite{reitzes2002rolling} in detail.
We find three mergers before 1998 and twelve mergers after the enactment of the OSRA of 1998.

Container shipping established the status as one of the global standards for carrying cargo in the 1990s. 
As a remarkable feature, new ports in east Asian countries were developed in the 1990s.
For example, Chinese ports’ entries in global shipping networks and the emergence of post-Panamax ships occurred \citep{GUERRERO2014151}. As a result, Several ports grew as new transshipment hubs like Salalah and Colon to accommodate growth in emerging economies like Vietnam, India and Brazil.

Global alliances in the container shipping were born in the 1990s \citep{Hirata2017}. If a shipping company decides to merge with another company when expanding its liner networks, it can have more market power, while it owns or charters a larger number of vessels, then is exposed to more huge risks of volatility in container freight or cargo volume. On the other hand, if it decides to form an alliance, it can offer customers more comprehensive networks without merging with other companies, they cannot have price-setting power because members are involved in the pricing decision separately. 
Shipping alliances involved cooperation on a global scale, mainly on trunk routes such as the transpacific and Far East to Europe, unlike earlier shipping conferences which were limited to specific routes. 
In 1994, MOL formed The Global Alliance (TGA) with APL, Nedlloyd and OOCL.
The following year, the Grand Alliance (GA) was formed by NYK, Hapag-Lloyd, NOL and P\&O Containers.
The CKY Alliance was formed in 1996 by KLINE, Cosco and Yang Ming Shipping. 
Hanjin Shipping entered to the CKY alliance in 2001, making the alliance CKYH.
At the same time, mergers were taking place, with Royal Nedlloyd Lines merging with P\&O in 1997.
Both companies became P\&O Nedlloyd and decided to join GA, and NOL, after acquiring APL, formed The New World Alliance (TNWA) with MOL and Hyundai Merchant Marine in 1998. 
Thus, firms that chose not to join alliances sought to expand through mergers, for example, CMA's acquisition of CGM in 1999 and Maersk's acquisition of Sea-Land and others.

In the first half of the 2000s, merger incentives decreased because many firms achieved substantial profits in the transpacific and Asia-Europe markets without resorting to mergers. This was attributed to factors such as increased exports of electrical appliances, furniture, household goods, and more, driven by China's rapid economic growth, a stable housing market in the U.S., and strong economic growth in Europe. In the first half of the 2000s, cargo movements, especially on routes from China to the U.S. and Europe, experienced significant growth.
As some exceptions, mergers and acquisitions were mainly European companies seeking to increase their scale, including Maersk's acquisition of P\&O Nedlloyd and Hapag-Lloyd's acquisition of CP Ships in 2005, and CMA-CGM's acquisition of Delmas in 2006.



\paragraph{2009-2022}

Table \ref{tb:merger_list_HB} summarizes all mergers based on IHS between 2006 and 2022. 
We find inconsistent merger cases between the HB data and institutional background so we split the merger cases into two categories.
In panel (a), we summarize the merger cases from the HB data consistent with the institutional history.
In panel (b), we summarize three merger cases from the HB data inconsistent with the institutional history.
The inconsistency comes from the fact that merged firms can keep the past operator name in the operation. 
We treat the three merger cases as occurring in the data according to the data records.
Summarizing the two panels, we find six mergers before 2009 and nineteen mergers after the enactment of the OSRA of 1998.

The remarkable feature of this period is the oversupply of shipping services.
After the U.S. housing market collapsed in 2007 with the revelation of the subprime mortgage crisis among financial institutions, transport volume growth slowed. 
Then, following the bankruptcy of Lehman Brothers the following year, the shipping demand for containerized cargo began to decline. 
On the other hand, along with the increase in the number of vessels, the size of the vessels also increased in order to reduce the transportation cost per container. 
This encouraged the expansion of service supply and deteriorated the supply-demand balance year by year. 
For example, \cite{matsuda2022} show that, compared with 1986, the volume of containerized cargo transport was 723\% in 2007 and 1047\% in 2016. On the other hand, the tonnage of container ships increased from 944\% in 2007 to 1784\% in 2016.
The oversupply of shipping services was the root cause of the market downturn in the late 2010s and was associated with falling freight rates to a competitive level.\footnote{Falling freight rates also led to alliance restructuring and mergers and acquisitions. In 2012, GA and TNWA began joint operations on Asia-Europe routes, forming the G6 alliance. The three European companies, Maersk, MSC, and CMA-CGM, have had a vessel-sharing agreement on transpacific routes since 2008 and had also been strengthening their alliance among the three companies. In 2013, the three companies announced the formation of a new alliance called the P3 Network. However, this was terminated due to lack of approval from the Chinese Ministry of Commerce. This was due to the fact that it would have had too large a market share on the Far East-Europe route. Following the failure of the P3 alliance, Maersk and MSC immediately signed a 10-year vessel-sharing agreement to form the 2M alliance. The remaining CMA-CGM also announced the formation of Ocean Three (O3) with China Shipping Container Line (CSCL) and Arab-owned United Arab Shipping Company (UASC), and began partnering on key routes in 2015. The CKYH alliance was also joined by Evergreen in April 2014, replacing the CKYHE alliance.} As a result, container shipping firms chose to merge in order to survive the tough shipping market.

Mergers and acquisitions have also made significant progress since 2014. In that year, Hapag-Lloyd's acquisition of CSAV, Hamburg Sud's acquisition of CCNI, and CMA-CGM's acquisition of German shipping line ODPR were announced. In 2015, CMA-CGM's acquisition of NOL and the acquisition of the container shipping divisions of COSCO Group and China Shipping Group. In 2016, Hapag-Lloyd's acquisition of UASC and Maersk's acquisition of Hamburg Sud were announced, and in Japan, NYK, MOL and KLINE announced the integration of their liner shipping divisions in 2016. The end of August 2016 also saw the first bankruptcy of a major shipping company (Hanjin Shipping) since the formation of alliances. Restructuring in the 2010s has been significant in scale, with the number of eight container shipping companies were bankrupt or merged between 2015 and 2018.
% Background
% What mergers?




% (Need to validate with NYK aggregate TEU data just for confirmation.)

\subsection{Descriptive statistics}\label{sec:descriptive_statistics}

Table \ref{tb:summary_statistics} presents summary statistics for firm-year-level variables of buyer and seller firms in all realized merger cases spanning 1966 to 2022. 
These variables include the firm's age in the global container shipping industry and size measured by TEU in the merger year. 
We normalize these values from 1e-6 (minimum age or size) to 1 (maximum age or size) within each regime for inter-regime comparison.
First, we observe a decline in the mean of normalized firm ages, from 0.84 in the 1966-1990 period to 0.52 in the 2006-2022 period. 
This suggests that recent mergers tend to involve relatively younger firms, that is, firms with less experience in the container shipping industry.

Second, the mean of normalized firm sizes decreases from 0.23 in the 1966-1990 period to 0.07 in the 2006-2022 period.
This indicates that recent mergers tend to involve relatively smaller firms, despite the exponential growth in firm size, particularly from 2006 to 2022.
It's worth noting that we treat new entrant firms purchasing incumbent firms as having an age and size of zero, reflecting the increased number of entries through mergers.
Figure \ref{fg:size_cdf} illustrates the cumulative distributions of normalized firm size and age for each regime, reinforcing the observation that recent mergers tend to feature younger and smaller firms.


\begin{table}[!htbp]
  \begin{center}
      \caption{Summary statistics of firm-year-level variables}
      \label{tb:summary_statistics} 
      \subfloat[CIY (1966-1990)]{
\begin{tabular}[t]{lrrrrr}
\toprule
  & N & mean & sd & min & max\\
\midrule
Age (Normalized) & 16 & 0.84 & 0.24 & 0.23 & 1.00\\
Size TEU (Normalized) & 16 & 0.23 & 0.29 & 0.01 & 1.00\\
\bottomrule
\end{tabular}
}\\
      \subfloat[IHS (1991-2005)]{
\begin{tabular}[t]{lrrrrr}
\toprule
  & N & mean & sd & min & max\\
\midrule
Age (Normalized) & 34 & 0.56 & 0.40 & 0.00 & 1.00\\
Size TEU (Normalized) & 34 & 0.13 & 0.24 & 0.00 & 1.00\\
\bottomrule
\end{tabular}
}\\
      \subfloat[HB (2006-2022)]{
\begin{tabular}[t]{lrrrrr}
\toprule
  & N & mean & sd & min & max\\
\midrule
Age (Normalized) & 60 & 0.51 & 0.30 & 0.11 & 1.00\\
Size TEU (Normalized) & 60 & 0.08 & 0.19 & 0.00 & 1.00\\
\bottomrule
\end{tabular}
}
      
  \end{center}\footnotesize
  %Note:
\end{table} 

\begin{figure}[!ht]
\begin{center}
  \includegraphics[width = 0.45\textwidth]
  {figuretable/normalized_size_cdf.png}
  \includegraphics[width = 0.45\textwidth]
  {figuretable/normalized_age_cdf.png}
  \caption{Distributions of firm-year-level variables for each regime}
  \label{fg:size_cdf}
  \end{center}
\footnotesize
   %Note: 
\end{figure}

\begin{figure}[!ht]
\begin{center}
  \includegraphics[width = 0.7\textwidth]
  {figuretable/distance_cdf.png}
  \caption{Distributions of match-level distances of seller and buyer firms for each regime}
  \label{fg:distance_cdf}
  \end{center}
\footnotesize
   %Note: 
\end{figure}


Figure \ref{fg:distance_cdf} depicts distributions of realized match-level distances.
We observe that nearly all mergers between 1966 and 1990 involve buyer and seller firms within the same country. 
Although mergers within the same country account for over 30\% of the entire period, the number of mergers between firms separated by distance has increased in recent years. 
This shift suggests that the reasons for mergers have transitioned from domestic to global, driven by the growth and expansion of international container transport.

These visualizations provide initial insights into the involvement of relatively younger and smaller firms in more distant countries in recent mergers. 
However, comprehensively assessing the relative importance of each variable with limited data requires a sophisticated structural model. 
In Section \ref{sec:empirical_analysis}, we introduce a structural matching model to quantify the significance of each variable and analyze the transitions across regimes.




\section{Empirical analysis}\label{sec:empirical_analysis}

Our objective is to quantify the assortativeness of observed characteristics for each regime and compare the levels across regimes. 
We employ a matching maximum score estimator developed by \cite{fox2018qe}, which is one of the most well-known methods for measuring matching assortativeness.

We model mergers in each regime as a two-sided one-to-one transferable matching game in a single market. Let $\mathcal{N}_b$ and $\mathcal{N}_s$ be the sets of potential finite buyers and sellers respectively. Let $b=1,\cdots,|\mathcal{N}_b|$ be buyer firms and let $s=1,\cdots,|\mathcal{N}_s|$ be seller firms where $|\cdot|$ is cardinality. Let $\mathcal{N}_{b}^{m}$ denote the set of ex-post matched buyers and $\mathcal{N}_{b}^{u}$ denote that of ex-post unmatched buyers such that $\mathcal{N}_b= \mathcal{N}_{b}^{m}\cup\mathcal{N}_{b}^{u}$ and $\mathcal{N}_{b}^{m}\cap\mathcal{N}_{b}^{u}=\emptyset$. For the seller side, define $\mathcal{N}_{s}^{u}$ and $\mathcal{N}_{s}^{m}$ as the set of ex-post matched and unmatched sellers such that $\mathcal{N}_s= \mathcal{N}_{s}^{m}\cup\mathcal{N}_{s}^{u}$ and $\mathcal{N}_{s}^{m}\cap\mathcal{N}_{s}^{u}=\emptyset$. Let $\mathcal{M}^m$ be the sets of all ex-post matched pairs $(b,s)\in\mathcal{N}_{b}^{m}\times \mathcal{N}_{s}^{m}$. Let $\mathcal{M}$ denote the set of all ex-post matched pairs $(b,s)\in\mathcal{M}^{m}$ and unmatched pairs $(\tilde{b},\emptyset)$ and $(\emptyset,\tilde{s})$ for all $\tilde{b}\in \mathcal{N}_b^u$ and $\tilde{s}\in \mathcal{N}_s^u$ where $\emptyset$ means a null agent generating unmatched payoff. 

Each firm can match at most one agent on the other side, so  $|\mathcal{N}_b^{m}|=|\mathcal{N}_s^{m}|$. The matching joint production function is defined as $f(b,s)=V_b(b,s)+V_s(b,s)$ where $V_b:\mathcal{M}\rightarrow \mathbb{R}$ and $V_s:\mathcal{M}\rightarrow \mathbb{R}$. The net matching values for buyer $b$ and seller $s$ are defined as $V_b(b,s)=f(b,s)-p_{b,s}$ and $V_s(b,s)+p_{b,s}$, where $p_{b,s}\in \mathbb{R}_{+}$ is the equilibrium merger price paid to seller firm $s$ by buyer firm $b$ and $p_{b\emptyset}=p_{\emptyset s}=0$. For scale normalization, we assume $V_b(b,\emptyset)=0$ and $V_s(\emptyset,s)=0$ for all $b\in \mathcal{N}_b$ and $s\in \mathcal{N}_s$. Each buyer maximizes $V_b(b,s)$ across seller firms, whereas each seller maximizes $V_s(b,s)$ across buyer firms. 

The stability conditions for buyer firm $b \in \mathcal{N}_b$ and seller firm $s \in \mathcal{N}_s$ are as follows:
\begin{align}
    V_b(b,s) &\ge V_b(b,s') \quad \forall s' \in \mathcal{N}_s \cup \emptyset,s'\neq s,\label{eq:stability_ineq}\\
    V_s(b,s) &\ge V_s(b',s) \quad \forall b' \in \mathcal{N}_b\cup \emptyset,b'\neq b.\nonumber
\end{align}

Based on Inequalities \eqref{eq:stability_ineq} and equilibrium price conditions $p_{b',s}\le p_{b,s}$ and $p_{b,s'}\le p_{b',s'}$ in \cite{akkus2015ms}, we construct the inequalities for matches $(b,s)\in \mathcal{M}$ and $(b',s')\in \mathcal{M}, (b',s')\neq(b,s)$ as follows:
\begin{align}
    f(b,s)-f(b,s')&\ge p_{b,s}-p_{b,s'}\ge p_{b,s}-p_{b',s'},\label{eq:pairwise_stable_ineq}\\
    f(b',s')-f(b',s)&\ge p_{b',s'}-p_{b',s}\ge p_{b',s'}-p_{b,s},\nonumber\\
    V_s(b,s)-V_s(b',s)&\ge 0,\nonumber\\
    V_{s'}(b',s')-V_s(b,s')&\ge 0,\nonumber
\end{align}
where $p_{b',s}$ and $p_{b,s'}$ are unrealized equilibrium merger prices that cannot be observed in the data. The last two inequalities cannot be derived from the data because the researchers cannot observe how the total matching value $f(b,s)$ is shared between buyer $b$ and seller $s$.

Each buyer firm can only acquire one seller firm, which implies that the buyer firm’s
choice among a set of seller firms is a discrete choice. 
As a simple semiparametric technique to estimate this discrete choice, we turn to maximum score estimation \cite{manski1975maximum,manski1985semiparametric}.
\cite{fox2018qe} proposes a maximum score
estimator using the above inequalities when we observe the transfer data or not under mild conditions. 
The maximum score estimator is consistent if the model satisfies a rank order property as in \cite{manski1975maximum,manski1985semiparametric}, i.e., the probability of observing matched pairs is larger than the probability of observing swapped matched pairs. 
The rank order property is equivalent to pairwise stability which is a milder property rather than stability, so that the rank order property holds under the above stability conditions. See identification details in \cite{fox2010qe} and Monte Carlo simulation results in \cite{fox2018qe}, \cite{akkus2015ms}, and \cite{otani2021matching_cost}.

We specify $f(b,s)$ as a parametric form $f(b,s|X,\beta)$ where $X$ is a vector of observed characteristics of all buyers and sellers and $\beta$ is a vector of parameters. 
In the absence of transfer data, given the observed characteristics $X$, one can estimate $\beta$ by maximizing the following objective function:
\begin{align}
    Q(\beta)=\sum_{(b,s)\in \mathcal{M}} \sum_{(b',s')\in \mathcal{M},(b',s')\neq (b,s)} \mathbbm{1}[f(b,s|X,\beta)+ f(b',s'|X,\beta)\ge f(b,s'|X,\beta)+f(b',s|X,\beta)]\label{eq:score_function}
\end{align}
where $\mathbbm{1}[\cdot]$ is an indicator function. 
The inequality is constructed by adding the first two inequalities and canceling out transfers $p_{b,s}$ and $p_{b',s'}$ in Inequalities \eqref{eq:pairwise_stable_ineq}.
The objective function \eqref{eq:score_function} counts the number of correctly predicted pairwise stable matching under each candidate parameter $\beta$.


In our empirical application, as the observed characteristics, $X$, we use the standardized firm's age, size measured by the total tonnage, and match-level distance calculated from the locations of the flag countries at merger timing, that is, all observed variables are standardized to $[1e-6,1]$. 
Concretely, we specify the joint production function $f(b,s)$ as
\begin{align}
    f(b,s|X,\beta)= \beta_1 \text{Age}_{b}\text{Age}_{s} + \beta_2 \text{Size}_{b}\text{Size}_{s} + \beta_3 \text{Distance}_{bs} + \varepsilon_{bs},\label{eq:joint_production}
\end{align}
where $\varepsilon_{bs}$ is assumed to be i.i.d. errors drawn from the zero median distribution as in \cite{fox2018qe}. 
Note that any parameters of firm-specific characteristics cannot be identified with maximum score estimation based solely on without-transfers information. With transfer data, $p_{b,s}$ and $p_{b',s'}$, such as the payments regarding mergers from buyer firms to seller firms, the identification is possible and the precision of the estimator improves as in \cite{akkus2015ms}.


\section{Results}\label{sec:results}

Table \ref{tb:maximum_score_estimate} reports the estimation results of the matching maximum score estimator. 
Since we can use only realized merger cases, we could not construct a 95 \% confidence interval with enough subsampled data via bootstrap.
Instead, the numbers in brackets indicate the lower and upper bounds of a set of maximizers of the objective function. 
If the lower and upper bounds are the same, then the parameters are point-identified. 
Otherwise, the parameters are partially identified.
The percent of correct matches used as a measure of statistical fit is more than 90\%, so the estimated model predicts the actual mergers well.

First, the estimated coefficient of the firm's size shows an interesting transition. 
The sign is changed from ambiguous between 1966 and 1990, positive between 1991 and 2005, to negative between 2006 and 2022. 
In particular, between 1991 and 2005, as a positive factor, firm's size is more important than firm's age by 9.974 times in merger decisions, that is, firm's size works as a merger incentive.
On the other hand, between 2006 and 2022, as a negative factor, firm's size is more important than firm's age by 0.02-0.63 times, that is, firm's size works as a merger disincentive.
These results are consistent with the institutional fact that consolidation-type mergers in which buyer firms have the lower bound of age and size variables at the initial merger timing have been common rather than absorption-type mergers in recent years.

Second, the estimated coefficient of the distance of seller and buyer firms shows a negative sign across all regimes but the level decreases between 2006 and 2022. 
This means that mergers of firms in distant countries are likely to occur, but the importance level relative to firm's age has decreased to economically zero in recent years.
These results are consistent with data patterns shown in Section \ref{sec:descriptive_statistics} and the institutional facts that shipping companies do not hesitate to merge with companies in distant regions to expand their container shipping networks.

\begin{table}[!htbp]
  \begin{center}
      \caption{Matching maximum score estimation}
      \label{tb:maximum_score_estimate} 
      
\begin{tabular}[t]{lcccl}
\toprule
Regime & 1966-1990 & 1991-2005 & 2006-2022 & Regime\\
 &  &  &  & \\
Firm age: $\beta_1$ &  & 1 & 1 & 1\\
Firm size (TEU): $\beta_2$ &  & {}[-9.5033,9.4747] & {}[7.1245,9.3379] & {}[-1.4208,-1.4208]\\
Distance: $\beta_3$ &  & {}[-9.9771,-0.6871] & {}[-3.2095,-2.7023] & {}[-0.0043,-0.0043]\\
 &  &  &  & \\
\% of correct matches &  & 1.0000 & 0.9316 & 0.9816\\
\bottomrule
\end{tabular}

  \end{center}\footnotesize
  Note: The objective function was numerically maximized using differential evolution (DE) algorithm in \texttt{BlackBoxOptim.jl} package. For the DE algorithm, we require setting the domain of parameters and the number of population seeds so that we fix the former to $[-10, 10]$. For estimation, 100 runs of 1,000 seeds were performed for all specifications. The numbers in parentheses are the lower and upper bounds of the set of maximizers of the maximum rank estimator. Parameters that can take on only a finite number of values (here 1) converge at an arbitrarily fast rate, then they are superconsistent. The unit of measure of all variables is normalized to $[1e-6,1]$. 
\end{table} 

\section{Counterfactual}\label{sec:counterfactuals}

In the global container shipping industry, mergers between firms in the same country involve several concerns from competition policies in multiple countries.
For example, on June 21, 2017, South Africa's Competition Commission issued a statement stating that it "forbade" the integration of the container business by the three shipping lines of NYK, MOL, and KLINE. 
The commission cited concerns about market consolidation by domestic companies and cartel issues involving these companies in the car carrier business.
The country's competition court finally approved the integration on January 17, 2018, but this could impact the planned launch of the integrated container company, Ocean Network Express. 
In another example, Cosco and China Shipping`s alliance decision in 2015 could face scrutiny from regulators as the market share in some east-west trades is likely to breach the 30\% mark if the new entity joins the group. Above 30\%, alliances must ensure their agreements comply with EU rules outlawing anti-competitive behavior.


In our counterfactual simulation, given estimated parameters, we simulate the matching outcome under the hypothetical scenario that mergers between firms in the same country are prohibited. 
Mechanically, first, this scenario uses the merger cases of firms in different countries and imposes that 
the joint production function \eqref{eq:joint_production} is changed to
\begin{align*}
    f(b,s|X,\beta)= \begin{cases}
        \beta_1 \text{Age}_{b}\text{Age}_{s} + \beta_2 \text{Size}_{b}\text{Size}_{s} + \beta_3 \text{Distance}_{bs} + \varepsilon_{bs}, \quad \text{if }\text{Distance}_{bs}\neq 1e-6,\\
        -\infty, \quad \text{otherwise}.
    \end{cases}
\end{align*}
Second, given counterfactual $f(b,s|X,\beta)$, we compute an equilibrium matching allocation. 
The equilibrium one-to-one matching allocation $\{m(b,s)\}_{b\in\mathcal{N}_b,s\in\mathcal{N}_s}$ is calculated by the following linear programming problem proposed by \cite{shapley1971assignment}:
\begin{align*}
    \max_{\{m(b,s)\}_{b\in\mathcal{N}_b,s\in\mathcal{N}_s}} &f(b,s|X,\beta)\cdot m(b,s),\\
    \text{s.t. } 0&\le \sum_{b\in\mathcal{N}_b}m(b,s)\le 1\quad  \forall s \in \mathcal{N}_s,\\
    0&\le \sum_{s\in\mathcal{N}_s}m(b,s)\le 1\quad \forall b \in \mathcal{N}_b,\\
    0&\le m(b,s) \quad \forall b \in \mathcal{N}_b,\forall s \in \mathcal{N}_s,
\end{align*}
where the dual of this linear programming problem also gives equilibrium prices. 
In the equilibrium matching allocation, $m(b,s) = 1$ if firms $b$ and $s$ are matched and $m(b,s) = 0$ otherwise.
In the simulation, we fix the parameters to the upper bounds of estimated parameters in Table \ref{tb:maximum_score_estimate}, and we draw 100 i.i.d. draws of $\varepsilon_{bs}$ from the standard normal distribution $N(0,1)$, then solve the above linear programming problem.
We confirm that using the lower bounds of parameters gives the same results.
We report the lower and upper bounds of percentages of the number of total matchings and the same matching configurations of the simulated 100 matching outcomes relative to the data.

\begin{table}[!htbp]
  \begin{center}
      \caption{Counterfactual simulations under the prohibition of mergers of firms in the same country}
      \label{tb:number_of_mergers_counterfactual} 
      
\begin{tabular}[t]{lccc}
\toprule
Regime &  & 1991-2005 & 2006-2022\\
\midrule
Matching Num (data) &  & 17 & 30\\
\% total match (counterfactual/data) &  & {}[1.000,1.000] & {}[1.000,1.000]\\
\% same match (counterfactual/data) &  & {}[0.133,0.500] & {}[0.133,0.500]\\
\bottomrule
\end{tabular}

  \end{center}\footnotesize
  Note: We simulate the matching outcome from the upper bounds of the estimated parameters and 100 i.i.d. draws of $\varepsilon_{bs}$ from the standard normal distribution $N(0,1)$. The bracket gives the lower and upper bounds of percentages of the number of total matchings and the same matching configurations of the simulated 100 matching outcomes relative to the data.
\end{table} 

Table \ref{tb:number_of_mergers_counterfactual} reports the counterfactual simulation results under the prohibition of mergers between firms in the same country. 
We compare hypothetical merger cases with actual cases and focus on the two regimes between 1991 and 2005 and between 2006 and 2022 since only one merger of firms in different countries happened between 1966 and 1990.
First, the counterfactual model predicts the number of mergers between firms in different countries perfectly in both regimes. 
Second, as an interesting finding, only 0.12 to 0.44\% of the counterfactual merger pairs are the same as actual pairs.
Intuitively, this implies that the prohibition would restrict the choice set of buyer firms involved in the prohibited mergers to the set of firms in different countries and change the matching outcome through maximization of the modified objective function \eqref{eq:score_function}.
Therefore, we find that the prohibition of mergers between firms in the same country affects the merger configuration of all firms in the industry.




% \section{Interviews}\label{sec:interviews}

% \textcolor{blue}{Concerning the transition of merger incentives from 1966 to 2022, Mr.
% Akimitsu Ashida, former chairman of Mitsui O.S.K. Lines (MOL), answered about the situation in those days. He served as the company’s European Division Manager from 1985-86 and responded
% to our e-mail inquiry on XXX. We asked him about his thoughts and related memory on the figures, tables, and our results.}



\section{Practical Implications, Discussion, and Future Research}\label{sec:practical_implications}

\subsection{Practical Implications}

Our study makes an important contribution to the development of the unified merger list of the container shipping industry and policy discussions for practitioners. 
The data contribution enables practitioners to obtain empirical and historical knowledge on the main container transport markets, as well as to develop a methodology to disentangle merger incentives.
For instance, our analysis corroborates anecdotal evidence indicating that container shipping firms that abstained from joining alliances in the 1980s, and had relatively brief market tenure, primarily engaged in mergers to increase their scale in their home regions. 
In recent years, they have tended to merge with companies in distant regions, even when there are no significant differences in size, to expand their container shipping networks.
Also, \cite{jeon2022learning} simulates counterfactual and exogenous merger scenarios of specific two firms between 2006 and 2014 but does not incorporate and explain more than ten merger cases in her model.
Thus, our data contribution complements previous studies institutionally and sheds light on the data patterns and main drivers of mergers in the industry.

As a contribution to the policy discussion, our counterfactual finds that the prohibition of mergers between firms in the same country would change the merger configuration significantly. 
This predicts the propagated impact of the prohibition caused by the local competition policies not only on the local markets but also on the global market configuration. 
For example, if Japanese container shipping companies were restricted from merging their container divisions due to a dominant market share in Japan, they might have pursued mergers with firms in other countries to enhance their shipping network and scale. Such a scenario could have altered the current structure of the container shipping market.

\subsection{Discussion}
We summarize the potential concerns for the data and methodology. 
First, we merge three data sources that record potentially different variables and observations for each regime. 
Thus, we could not check the robustness check on the choice of the regime due to data limitation and inconsistency, although we believe that our choice of the regime is reasonable for institutional and graphical reasons.
Second, we may face a small sample problem, in particular, between 1966 and 1990 even though the matching maximum score approach works in a small sample in Monte Carlo simulations \citep{akkus2015ms,otani2021matching_cost}.
Third, we drop merger cases that involve firms whose variables are missing in our three data sources. 
This might be due to the fact that, unlike the MDS Transmodal data mentioned above, it does not cover all of the routes and vessels in the container trade. This is because the IHS data lacks some historical information on vessel operations, and the HB data collects data on routes and vessels focusing on trunk lines and routes to/from Japan, using operator advertisements. For example, Maersk's merger with Sea-Land and P\&O Containers' merger with Nedlloyd are not in the data and are not reflected in the analysis, as shown in the footnotes in the merger lists in Tables \ref{tb:merger_list_IHS} and \ref{tb:merger_list_HB}.


\subsection{Future Research}

We should discuss possible extensions as well as some shortcomings of this study. 
As a methodological issue, first, this paper focuses on disentangling endogenous merger incentives while ignoring future competition in the market due to data limitations. 
Thus, a welfare evaluation of the post-merger market was not investigated.
Combining firms' strategic interactions with estimations of demand and supply sides with the endogenous matching merger model remains a challenging and open research question in the field of industrial organization \citep{agarwal2021market}. 
An exceptional study is \cite{igami2019mergers} which construct a stochastic sequential-move game but need monthly-level merger data and allow only a single merger each month. 
Developing their approach might resolve the relationship between matching and competition.
Second, our matching model does not incorporate unobserved heterogeneity which is identified nonparametrically \citep{fox2018jpe}.
Pursuing this direction will require a different econometric approach such as the simulated method of moments in \cite{fox2018jpe} and multiple market data. 





\section{Conclusion}\label{sec:conclusion}
We construct a novel unified list of mergers in the global container shipping industry between 1966 (the beginning of the industry) and 2022. 
Combining the list with proprietary data, we construct a structural matching model to describe the historical transition of the importance of firm's age, size, and geographical proximity on merger decisions. 
We find different transition patterns of the importance of firm's age, size, and geographical proximity.
In counterfactual simulations, we find that the prohibition of mergers between firms in the same country affects the merger configuration of not only firms involved in prohibited mergers but also firms involved in permitted mergers.

\newpage

\textbf{Acknowledgement} \\
\textcolor{blue}{We benefited from anonymous referees and participants at the XXX. We thank Akimitsu Ashida and Hiroyuki Sato for sharing industry knowledge and expertise as ex-chairperson in the 1980s and Jeremy Fox for introducing methodology. And we thank Mikio Tasaka, Yasuhiro Fujita, Jong-khil Han and Yutaka Yamomoto for professional comments. This study was supported by JSPS KAKENHI Grant Numbers 20K22129 and 22K13501. }



\bibliographystyle{ecca}
\bibliography{container_merger_data_bib}


\end{document}