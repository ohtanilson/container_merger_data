\documentclass[10pt]{article}
\usepackage[utf8]{inputenc}
\usepackage{amsmath,setspace,geometry}
\usepackage{amsfonts}
\usepackage[shortlabels]{enumitem}
%\usepackage[dvipdfmx]{hyperref,graphicx}
\usepackage{graphicx}
\usepackage{bbm}

\usepackage[colorlinks,citecolor=purple,urlcolor=blue,bookmarks=false,hypertexnames=true]{hyperref}
\usepackage[]{natbib} 
\bibpunct[:]{(}{)}{,}{a}{}{,}
\geometry{left = 1.0in,right = 1.0in,top = 1.0in,bottom = 1.0in}
%\onehalfspacing
% \usepackage{setspace}
\doublespacing
%\renewcommand{\baselinestretch}{0.3}
\usepackage[english]{babel}
\usepackage{float}
\usepackage{subfig}
\usepackage{booktabs}
\usepackage{pdfpages}
\usepackage{threeparttable}
\usepackage{lscape}
%\setstretch{1.2}
\newtheorem{assumption}{Assumption}
\newtheorem{definition}{Definition}
\newtheorem{example}{Example}
\newtheorem{lemma}{Lemma}

\title{Unified Dataset of Entry, Exit, and Mergers in the Container Shipping Industry between 1966 and 2022}
\author{Takuma Matsuda\thanks{Faculty of Commerce, Takushoku University. Email: tmatsuda@ner.takushoku-u.ac.jp}\quad Suguru Otani\thanks{Department of Economics, Rice University. Email: so19@rice.edu}}
\date{
First version: XXX\\
Current version: \today
}

\begin{document}

\maketitle

\begin{abstract}
We construct a new unified panel dataset that records entry, exit, and mergers in the shipping industry between 1966 and 2022.
\end{abstract} 

\vspace{0.1in}
\noindent\textbf{Keywords:} container shipping industry; exemption agreement; container crisis; 
\vspace{0in}


\section{Introduction}


\section{Industry Background}
We describe industry background between 1966 and 2022 in chronological order by focusing on firms' entry, exit, and mergers. 


\paragraph{1966-1979} 


\cite{matsuda2022unified}

\paragraph{1980-1990}

\cite{matsuda2022unified}

\paragraph{1991-2008}

\paragraph{2009-2022}

\section{Data}

We construct the data from combining the three data sources. 
The first data source is \textit{the Containerization International Yearbook} (CIY), which provides ship-level information between 1966 and 1990.
The second data source is IHS data, which provides ship-level information between 1991 and 2005.
The third data source is \textit{Handbook of Ocean Commerce}, which provides ship-level information between 2006 and 2022.

We define the entry of the firm as the existence of its owned container ships in the industry. 
Also, we define the exit of the firm as the liquidation or bankrupt of the firms or being merged by the other firm. 

\begin{table}[!htbp]
  \begin{center}
      \caption{Merger list}
      \label{tb:merger_list_CIY} 
      \subfloat[CIY (1966-1990)]{
\begin{tabular}[t]{rllrl}
\toprule
ID & Seller & Buyer & Year & Type\\
\midrule
1 & Moore-McCormack Lines Inc & United States Lines & 1970 & acquisition\\
2 & OCL & P\&O Containers & 1986 & merger\\
3 & Franco-Belgian Services & Maersk & 1986 & merger\\
4 & Y-S Line & NLS & 1988 & merger\\
5 & Japan Line & NLS & 1988 & merger\\
6 & KSC & Hanjin & 1988 & merger\\
7 & Finland Steamship & Finnlines & 1990 & merger\\
8 & Atlanttrafik/Barber Blue Sea & Wilhelmsen Lines A/S & 1990 & merger\\
\bottomrule
\end{tabular}
}\\
      \subfloat[HIS (1991-2005)]{
\begin{tabular}[t]{rllrl}
\toprule
ID & Seller & Buyer & Year & Type\\
\midrule
1 & Moore-McCormack Lines Inc & United States Lines & 1970 & acquisition\\
2 & OCL & P\&O Containers & 1986 & merger\\
3 & Franco-Belgian Services & Maersk & 1986 & merger\\
4 & Y-S Line & NLS & 1988 & merger\\
5 & Japan Line & NLS & 1988 & merger\\
6 & KSC & Hanjin & 1988 & merger\\
7 & Finland Steamship & Finnlines & 1990 & merger\\
8 & Atlanttrafik/Barber Blue Sea & Wilhelmsen Lines A/S & 1990 & merger\\
\bottomrule
\end{tabular}
}\\
      \subfloat[HB (2005-2022)]{
\begin{tabular}[t]{rllrl}
\toprule
ID & Seller & Buyer & Year & Type\\
\midrule
1 & Cheng Lie & CMA-CGM & 2006 & acquisition\\
2 & Lloyd Triestino & Evergreen & 2006 & merger\\
3 & Norasia & CSAV & 2006 & acquisition\\
4 & MacAndrews & CMA-CGM & 2007 & acquisition\\
5 & Lufeng & Sinotrans & 2008 & merger\\
6 & NEW ONTO SHIPPING & GOTO Shipping International Ltd & 2010 & merger\\
7 & TSK & NYK & 2010 & merger\\
8 & China Navigation & Swire & 2011 & acquisition\\
9 & CCNI & Maersk & 2015 & acquisition\\
10 & CSAV & Hapag-Lloyd & 2015 & acquisition\\
11 & China Shipping & COSCO & 2016 & merger\\
12 & Shanghai Puhai Shipping & COSCO & 2016 & merger\\
13 & UASC & Hapag-Lloyd & 2017 & acquisition\\
14 & KLINE & Ocean Network Express & 2018 & merger\\
15 & MOL & Ocean Network Express & 2018 & merger\\
16 & NYK & Ocean Network Express & 2018 & merger\\
17 & APL & CMA-CGM & 2017 & acquisition\\
18 & Hamburg Sud & Maersk & 2018 & acquisition\\
\bottomrule
\end{tabular}
}
  \end{center}\footnotesize
  Note: 
\end{table} 

Need to validate with NYK aggregate TEU data.

\section{Empirical model}

We investigate what types of firms survive the industry.

\section{Results}

\section{Conclusion}

\bibliographystyle{aer}
\bibliography{container_merger_data_bib}


\end{document}