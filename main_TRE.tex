\documentclass[10pt]{article}
\usepackage[utf8]{inputenc}
\usepackage{amsmath,setspace,geometry}
\usepackage{amsfonts}
\usepackage[shortlabels]{enumitem}
%\usepackage[dvipdfmx]{hyperref,graphicx}
\usepackage{graphicx}
\usepackage{bbm}

\usepackage[colorlinks,citecolor=purple,urlcolor=blue,bookmarks=false,hypertexnames=true]{hyperref}
\usepackage[]{natbib} 
\bibpunct[:]{(}{)}{,}{a}{}{,}
\geometry{left = 1.0in,right = 1.0in,top = 1.0in,bottom = 1.0in}
%\onehalfspacing
% \usepackage{setspace}
\doublespacing
%\renewcommand{\baselinestretch}{0.3}
\usepackage[english]{babel}
\usepackage{float}
\usepackage{subfig}
\usepackage{booktabs}
\usepackage{pdfpages}
\usepackage{threeparttable}
\usepackage{lscape}
%\setstretch{1.2}
\newtheorem{assumption}{Assumption}
\newtheorem{definition}{Definition}
\newtheorem{example}{Example}
\newtheorem{lemma}{Lemma}

\title{Unified Dataset of Mergers in the Container Shipping Industry between 1966 and 2022: Transition of Importance of Tonnage Capacity and Geographical Proximity on Merger Decision}
\author{Suguru Otani\thanks{Department of Economics, Rice University. Email: so19@rice.edu}\quad  Takuma Matsuda\thanks{Faculty of Commerce, Takushoku University. Email: tmatsuda@ner.takushoku-u.ac.jp}}
\date{
First version: XXX\\
Current version: \today
}

\begin{document}

\maketitle

\begin{abstract}
We construct a novel unified panel dataset of mergers in the shipping industry between 1966 and 2022. Using the data, we construct a structural matching model \citep{fox2018qe} to describe the historical transition of the importance of tonnage capacity and geographical proximity on merger decisions. We find that XXX.
\end{abstract} 

\vspace{0.1in}
\noindent\textbf{Keywords:} container shipping industry; exemption agreement; container crisis; 
\vspace{0in}


\section{Introduction}


\subsection{Related literature}

This paper contributes to three strands of the literature, namely, empirical transferable utility (TU) matching, the analysis of endogenous mergers, and recent industrial policy and antitrust studies in the shipping industry.

First, this paper contributes to the literature on empirical TU matching. 
The most related econometric model is \cite{fox2018qe}, whose model has been applied to other empirical topics such as banking merger \citep{akkus2015ms,chen2013ijio}, faculty room allocation \citep{baccara2012aer}, executive and firm matching \citep{pan2017determinants}, and buyer and seller relationships in the broadcast television industry \citep{stahl2016aer}. 
These papers have applied the matching maximum score estimator proposed by \cite{fox2010qe,fox2018qe} to two-sided many-to-many and one-to-one matching in a TU matching environment. 
We apply the approach to merger waves in the global container shipping industry from its inception.

Second, this paper contributes to the literature on endogenous merger analysis. Endogenous merger analysis in the industrial organization literature is divided into dynamic and static matching models. 
In terms of dynamic matching models, they follow \cite{gowrisankaran1999dynamic}.\footnote{\cite{stahl2011dynamic} was the first to estimate a merger activity model using a dynamic, strategic framework. \cite{jeziorski2014effects} estimated the sequential merger process to analyze ownership consolidation in the United States radio industry after the enactment of the Telecommunications Act of 1996. \cite{igami2019mergers} applied a stochastic sequential bargaining model to the merger processes of the hard disk industry. As the most recent paper, \cite{hollenbeck2020horizontal} enriched the Gowrisankaran-type dynamic endogenous merger model. With different dynamic approaches, \cite{nishida2015better} compared post-merger and pre-merger beliefs and equilibrium behaviors in a Markov perfect equilibrium in the Japanese retail chain industry. \cite{perez2015building} incorporated mergers as bidding games by incumbents and investigated the effect of the Reagan-Bush administration's merger policy on the reallocation of assets in the United States cement industry.} Conversely, using a static matching model, \cite{uetake2019entry} developed an empirical two-sided non-transferred utility matching model with externalities using moment inequalities and investigated the effect of entry deregulation on the ``with whom"-decisions of bank mergers by the Riegle-Neal Act. 
\cite{akkus2015ms} tackled the same Act with a different approach. 
They added transfer data and constructed a one-to-one matching model with transfer utility and found that merger value increased from cost efficiencies in overlapping markets, relaxing regulations, and the network effects exhibited by acquirer-target matching. 
Our paper follows \cite{akkus2015ms} and focuses on endogenous mergers in a single, static, large matching market for each regime. 
In addition, the present paper quantifies the relative importance of tonnage capacity and geographical proximity, which are the main economic forces driving firms to pursue mergers to gain cost efficiency in the shipping industry \citep{notteboom2004container}. 

Third, our paper contributes to the literature on recent industrial policy and antitrust studies in the shipping industry. \cite{jeon2022learning} studies the relationship between learning and investment in the container shipping industry between 2006 and 2014 and simulates counterfactual merger scenarios in which a merger occurred between top two firms that jointly account for over 35\% of total capacity in the industry.
\textcolor{blue}{[XXX]}



\section{Data}

We construct the data from combining the three data sources. 
The first data source is \textit{the Containerization International Yearbook} (CIY), which provides ship-level information between 1966 and 1990.
The second data source is IHS Markit data (IHS), which provides ship-level information between 1991 and 2005.
The third data source is \textit{Handbook of Ocean Commerce} (HB), which provides ship-level information between 2006 and 2022. 
Aggregating the data, we construct firm-level variables such as country name and tonnage capacity measured by TEU. 
Finally, we manually collect institutional information and construct a merger list that contains buyer names, seller names, and merger year, then merge the list with the firm-year-level variables.

Table \ref{tb:summary_statistics} shows summary statistics of buyer and seller firms in all realized merger cases between 1966 and 2022. 
First, we find that the mean of normalized firms' age decreases from 0.84 in the 1966-1990 period to 0.46 in the 2005-2022 period. This implies that mergers are likely to occur between younger firms in the recent year.
Second, the mean of normalized firms' size is the highest in the 1966-1990 period, and it is stable in the 1991-2005 and 2005-2022 periods.

\begin{table}[!htbp]
  \begin{center}
      \caption{Summary statistics of buyer and seller firms}
      \label{tb:summary_statistics} 
      \subfloat[CIY (1966-1990)]{
\begin{tabular}[t]{lrrrrr}
\toprule
  & N & mean & sd & min & max\\
\midrule
Age (Normalized) & 16 & 0.84 & 0.24 & 0.23 & 1.00\\
Size TEU (Normalized) & 16 & 0.23 & 0.29 & 0.01 & 1.00\\
\bottomrule
\end{tabular}
}\\
      \subfloat[IHS (1991-2005)]{
\begin{tabular}[t]{lrrrrr}
\toprule
  & N & mean & sd & min & max\\
\midrule
Age (Normalized) & 40 & 0.58 & 0.39 & 0.00 & 1.00\\
Size TEU (Normalized) & 40 & 0.11 & 0.22 & 0.00 & 1.00\\
\bottomrule
\end{tabular}
}\\
      \subfloat[HB (2005-2022)]{
\begin{tabular}[t]{lrrrrr}
\toprule
  & N & mean & sd & min & max\\
\midrule
Age (Normalized) & 60 & 0.51 & 0.30 & 0.11 & 1.00\\
Size TEU (Normalized) & 60 & 0.08 & 0.19 & 0.00 & 1.00\\
\bottomrule
\end{tabular}
}
      
  \end{center}\footnotesize
  Note:
\end{table} 


Figure XXX illustrates distributions of realized buyer-seller-level distances.

These tables and figures provide some intuition that XXX. However, disentangling the relative importance of each variable needs a more sophisticated structural model.  


\section{Industry Background}
We describe the industry background between 1966 and 2022 in chronological order by focusing on firms' mergers. We classify the periods into three regimes, 1966-1990, 1991-2008, and 2009-2022. Each regime corresponds with the institutional background and data source.

\begin{table}[!htbp]
  \begin{center}
      \caption{Merger list: CIY (1966-1990)}
      \label{tb:merger_list_CIY} 
      
\begin{tabular}[t]{rllrl}
\toprule
ID & Seller & Buyer & Year & Type\\
\midrule
1 & Moore-McCormack Lines Inc & United States Lines & 1970 & acquisition\\
2 & OCL & P\&O Containers & 1986 & merger\\
3 & Franco-Belgian Services & Maersk & 1986 & merger\\
4 & Y-S Line & NLS & 1988 & merger\\
5 & Japan Line & NLS & 1988 & merger\\
6 & KSC & Hanjin & 1988 & merger\\
7 & Finland Steamship & Finnlines & 1990 & merger\\
8 & Atlanttrafik/Barber Blue Sea & Wilhelmsen Lines A/S & 1990 & merger\\
\bottomrule
\end{tabular}

  \end{center}\footnotesize
  Note: The firms not operated in the merged year are treated as the firms that have a constant capacity level from the last active year in the merged year. For example, Johnson Line was active in 1969-1972 but was merged in 1991.
\end{table} 

\begin{table}[!htbp]
  \begin{center}
      \caption{Merger list: IHS (1991-2005)}
      \label{tb:merger_list_IHS} 
      
\begin{tabular}[t]{llr}
\toprule
Seller & Buyer & Year\\
\midrule
IMC SHIPPING CO PTE LTD & IMC SHIPPING CO PTE LTD & 1993\\
BUSAN SHIPPING CO LTD & EUROSEAS LTD & 1994\\
CHINA MERCHANTS STEAM NAVIGATI & China Merchants Group & 1994\\
SVITZER AS & A P MOLLER & 1996\\
APL LTD & NEPTUNE ORIENT LINES LTD (NOL) & 1997\\
PRIMA SHIPMANAGEMENT SDN BHD & HALIM MAZMIN GROUP & 1999\\
FARRELL LINES INC & A P MOLLER & 2000\\
OOST ATLANTIC LIJN BV & ATLANTIC HORIZON GROUP & 2001\\
CYPRUS MARITIME CO LTD & CYPRUS SEA LINES SA & 2002\\
MISC BERHAD & Malaysia Shipping Corp Sdn Bhd & 2003\\
DANSK SUPERMARKED INVEST A/S & A P MOLLER & 2003\\
THE PENINSULAR AND ORIENTAL ST & A P MOLLER & 2004\\
EUROBULK LTD & EUROSEAS LTD & 2005\\
BARCLAY SHIPPING LTD & BARCLAY SHIPPING LTD & 2005\\
DELMAS & CMA CGM HOLDING & 2006\\
ROYAL P\&O NEDLLOYD NV & A P MOLLER & 2006\\
UNITED THAI SHIPPING CORP LTD & IMC SHIPPING CO PTE LTD & 2006\\
HORIZON LINES INC & MATSON NAVIGATION CO INC & 2006\\
EICKE SCHIFFAHRTS KG & EICKE SCHIFFAHRTS KG & 2006\\
CP SHIPS LTD & HAPAG-LLOYD AG & 2006\\
\bottomrule
\end{tabular}

  \end{center}\footnotesize
  Note: 
\end{table} 

\begin{table}[!htbp]
  \begin{center}
      \caption{Merger list: HB (2005-2022)}
      \label{tb:merger_list_HB} 
      
\begin{tabular}[t]{rllrl}
\toprule
ID & Seller & Buyer & Year & Type\\
\midrule
1 & Cheng Lie & CMA-CGM & 2006 & acquisition\\
2 & Lloyd Triestino & Evergreen & 2006 & merger\\
3 & Norasia & CSAV & 2006 & acquisition\\
4 & MacAndrews & CMA-CGM & 2007 & acquisition\\
5 & Lufeng & Sinotrans & 2008 & merger\\
6 & NEW ONTO SHIPPING & GOTO Shipping International Ltd & 2010 & merger\\
7 & TSK & NYK & 2010 & merger\\
8 & China Navigation & Swire & 2011 & acquisition\\
9 & CCNI & Maersk & 2015 & acquisition\\
10 & CSAV & Hapag-Lloyd & 2015 & acquisition\\
11 & China Shipping & COSCO & 2016 & merger\\
12 & Shanghai Puhai Shipping & COSCO & 2016 & merger\\
13 & UASC & Hapag-Lloyd & 2017 & acquisition\\
14 & KLINE & Ocean Network Express & 2018 & merger\\
15 & MOL & Ocean Network Express & 2018 & merger\\
16 & NYK & Ocean Network Express & 2018 & merger\\
17 & APL & CMA-CGM & 2017 & acquisition\\
18 & Hamburg Sud & Maersk & 2018 & acquisition\\
\bottomrule
\end{tabular}

  \end{center}\footnotesize
  Note:
\end{table} 

\paragraph{1966-1990} 

\cite{matsuda2022unified}

Table \ref{tb:merger_list_CIY} summarizes

In the period between 1966 and 1983, two mergers occurred in 1970 (Moore-McCormack Lines Inc merged by United States Lines) and 1972 (Johnson Line merged by NA).

In the period between 1984 and 1990, two mergers occurred in 1986, three mergers occurred in 1988, and two mergers occurred in 1990. 

% Background
% What mergers?



\cite{matsuda2022unified}

\paragraph{1991-2008}

% Background
% What mergers?

Table \ref{tb:merger_list_IHS}


\paragraph{2009-2022}

% Background
% What mergers?

Table \ref{tb:merger_list_HB}





% (Need to validate with NYK aggregate TEU data just for confirmation.)

\section{Empirical analysis}\label{sec:empirical_analysis}

Our objective is to quantify the assortativeness of observed characteristics and merger costs for each regime and compare the levels across regimes. We employ a matching maximum estimator developed by \cite{fox2018qe}, which is one of the most well-known methods for measuring matching assortativeness.

For exposition, we model mergers in each regime as a two-sided one-to-one transferable matching game in a single market. Let $\mathcal{N}_b$ and $\mathcal{N}_s$ be the sets of potential finite buyers and sellers respectively. Let $b=1,\cdots,|\mathcal{N}_b|$ be buyer firms and let $s=1,\cdots,|\mathcal{N}_s|$ be seller firms where $|\cdot|$ is cardinality. Let $\mathcal{N}_{b}^{m}$ denote the set of ex-post matched buyers and $\mathcal{N}_{b}^{u}$ denote that of ex-post unmatched buyers such that $\mathcal{N}_b= \mathcal{N}_{b}^{m}\cup\mathcal{N}_{b}^{u}$ and $\mathcal{N}_{b}^{m}\cap\mathcal{N}_{b}^{u}=\emptyset$. For the seller side, define $\mathcal{N}_{s}^{u}$ and $\mathcal{N}_{s}^{m}$ as the set of ex-post matched and unmatched sellers such that $\mathcal{N}_s= \mathcal{N}_{s}^{m}\cup\mathcal{N}_{s}^{u}$ and $\mathcal{N}_{s}^{m}\cap\mathcal{N}_{s}^{u}=\emptyset$. Let $\mathcal{M}^m$ be the sets of all ex-post matched pairs $(b,s)\in\mathcal{N}_{b}^{m}\times \mathcal{N}_{s}^{m}$. Let $\mathcal{M}$ denote the set of all ex-post matched pairs $(b,s)\in\mathcal{M}^{m}$ and unmatched pairs $(\tilde{b},\emptyset)$ and $(\emptyset,\tilde{s})$ for all $\tilde{b}\in \mathcal{N}_b^u$ and $\tilde{s}\in \mathcal{N}_s^u$ where $\emptyset$ means a null agent generating unmatched payoff. 

Each firm can match at most one agent on the other side, so  $|\mathcal{N}_b^{m}|=|\mathcal{N}_s^{m}|$. The matching joint production function is defined as $f(b,s)=V_b(b,s)+V_s(b,s)$ where $V_b:\mathcal{M}\rightarrow \mathbb{R}$ and $V_s:\mathcal{M}\rightarrow \mathbb{R}$. The net matching values for buyer $b$ and seller $s$ are defined as $V_b(b,s)=f(b,s)-p_{b,s}$ and $V_s(b,s)+p_{b,s}$, where $p_{b,s}\in \mathbb{R}_{+}$ is the equilibrium merger price paid to seller firm $s$ by buyer firm $b$ and $p_{b\emptyset}=p_{\emptyset s}=0$. For scale normalization, we assume $V_b(b,\emptyset)=0$ and $V_s(\emptyset,s)=0$ for all $b\in \mathcal{N}_b$ and $s\in \mathcal{N}_s$. Each buyer maximizes $V_b(b,s)$ across seller firms, whereas each seller maximizes $V_s(b,s)$ across buyer firms. 

The stability conditions for buyer firm $b \in \mathcal{N}_b$ and seller firm $s \in \mathcal{N}_s$ are as follows:
\begin{align}
    V_b(b,s) &\ge V_b(b,s') \quad \forall s' \in \mathcal{N}_s \cup \emptyset,s'\neq s,\label{eq:stability_ineq}\\
    V_s(b,s) &\ge V_s(b',s) \quad \forall b' \in \mathcal{N}_b\cup \emptyset,b'\neq b.\nonumber
\end{align}

Based on Equation \eqref{eq:stability_ineq} and equilibrium price conditions $p_{b',s}\le p_{b,s}$ and $p_{b,s'}\le p_{b',s'}$ in \cite{akkus2015ms}, we construct the inequalities for matches $(b,s)\in \mathcal{M}$ and $(b',s')\in \mathcal{M}, (b',s')\neq(b,s)$ as follows:
\begin{align}
    f(b,s)-f(b,s')&\ge p_{b,s}-p_{b,s'}\ge p_{b,s}-p_{b',s'},\label{eq:pairwise_stable_ineq}\\
    f(b',s')-f(b',s)&\ge p_{b',s'}-p_{b',s}\ge p_{b',s'}-p_{b,s},\nonumber\\
    V_s(b,s)-V_s(b',s)&\ge 0,\nonumber\\
    V_{s'}(b',s')-V_s(b,s')&\ge 0,\nonumber
\end{align}
where $p_{b',s}$ and $p_{b,s'}$ are unrealized equilibrium merger prices that cannot be observed in the data. The last two inequalities cannot be derived from the data because the researchers cannot observe how the total matching value $f(b,s)$ is shared between buyer $b$ and seller $s$.

\cite{fox2018qe} proposes a maximum score
estimator using the above inequalities. The maximum score estimator is consistent if the model satisfies a rank order property, i.e., the probability of observing matched pairs is larger than the probability of observing swapped matched pairs. We specify $f(b,s)$ as a parametric form $f(b,s|X,\beta)$ where $X$ is a vector of observed characteristics of all buyers and sellers and $\beta$ is a vector of parameters. As observed characteristics $X$, we use the age and total tonnage of each firm and the geological location of the flag country at merger timing.


Given $X$, one can estimate $\beta$ by maximizing the following objective function:
\begin{align}
    Q(\beta)=\sum_{(b,s)\in \mathcal{M}} \sum_{(b',s')\in \mathcal{M},(b',s')\neq (b,s)} \mathbbm{1}[f(b,s|X,\beta)+ f(b',s'|X,\beta)\ge f(b,s'|X,\beta)+f(b',s|X,\beta)]\label{eq:score_function}
\end{align}
where $\mathbbm{1}[\cdot]$ is an indicator function.


\section{Results}

Table \ref{tb:maximum_score_estimate}

\begin{table}[!htbp]
  \begin{center}
      \caption{Matching maximum score estimation}
      \label{tb:maximum_score_estimate} 
      
\begin{tabular}[t]{lcc}
\toprule
Regime & 1966-1990 & 2005-2022\\
Standardized Firm age & 1 & 1\\
Standardized Firm-year-level TEU & {}[3e-04,0.0331] & {}[0.0019,0.1197]\\
\% of correct matches & 0.9643 & 0.9715\\
\bottomrule
\end{tabular}

  \end{center}\footnotesize
  Note: 
\end{table} 

\section{Counterfactual}



\section{Conclusion}

\bibliographystyle{aer}
\bibliography{container_merger_data_bib}


\end{document}